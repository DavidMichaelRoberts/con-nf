In this chapter, we construct the ambient environment inside which our model will reside.
To do this, we will set up various pieces of abstract machinery that will help us later.
Some mathematical background not already in mathlib will be included in \cref{c:auxiliary}.

\section{Conventions}
\begin{itemize}
  \item We are working in Lean's type theory, so cardinals and ordinals are quotients of a large type.
  In particular, cardinals are not just specific ordinals, and types cannot be ordinals.
  \item We write \( \#\tau \) for the cardinality of a type \( \tau \).
  \item If \( \tau \) is a type endowed with a well-order \( < \), we write \( \ot(\tau) \) for the order type of \( \tau \) with this well-ordering.
  \item The initial ordinal corresponding to a cardinal \( c \) is denoted \( \ord(c) \).
  The cofinality of an ordinal \( o \) is \( \cof(o) \), and this is a cardinal.
  \item The symmetric difference of two sets is denoted \( s \symmdiff t \coloneq (s \setminus t) \cup (t \setminus s) \).
  Note that \( (s \symmdiff t) \cup (s \cap t) = s \cup t \), and the union on the left-hand side is disjoint.
  \item We use \( f[s] \) for the direct image \( \{ f(x) \mid x \in s \} \). We write \( f^{-1}[s] \) for the inverse image \( \{ x \mid f(x) \in s \} \), and \( f^{-1}(x) \) for the fibre \( \{ y \mid f(y) = x \} \).
\end{itemize}

\section{Model parameters}
\begin{definition}[model parameters]
  \label{def:Params}
  A collection of \emph{model parameters} is a tuple \( (\lambda, {<_\lambda}, \kappa, \mu) \) such that
  \begin{itemize}
    \item \( {<} = {<_\lambda} \) is a well-order on \( \lambda \), and under this ordering, \( \lambda \) has no maximal element (so \( \ot(\lambda) \) is a limit ordinal);
    \item \( \#\kappa \) is uncountable and regular;
    \item \( \#\mu \) is a strong limit, and satisfies
    \[ \#\kappa < \#\mu;\quad \#\kappa \leq \cof(\ord(\#\mu));\quad \ot(\lambda) \leq \ord(\cof(\ord(\#\mu))) \]
    so in particular, \( \ot(\lambda) \leq \ord(\#\mu) \).
    Note that the inequalities in \( \kappa \) are inequalities of cardinals; the inequality in \( \lambda \) is an inequality of ordinals.
  \end{itemize}
  Given a collection of model parameters, we define
  \begin{itemize}
    \item canonical well-orders on \( \kappa \) and \( \mu \) such that \( \ot(\kappa) = \ord(\#\kappa) \) and \( \ot(\mu) = \ord(\#\mu) \); and
    \item a canonical left-cancellative additive monoid on \( \kappa \), obtained by passing through the equivalence \( \kappa \simeq \{ o : \Ord \mid o < \ord(\#\kappa) \} \).
  \end{itemize}
\end{definition}
\begin{proposition}
  \label{prop:Params.minimal}
  \uses{def:Params}
  The tuple \( (\mathbb N, {<_{\mathbb N}}, \aleph_1, \beth_{\omega_1}) \) is a collection of model parameters, where the symbols \( \aleph_1 \) and \( \beth_{\omega_1} \) represent particular types of that cardinality.
\end{proposition}
\begin{definition}[type index]
  \label{def:TypeIndex}
  \uses{def:Params}
  The type of \emph{type indices} is \( \lambda^\bot \coloneq \texttt{WithBot}(\lambda) \): the collection of \emph{proper type indices} \( \lambda \) together with a designated symbol \( \bot \) which is smaller than all proper type indices.
  Note that \( \ot(\lambda^\bot) = \ot(\lambda) \), and hence that for each \( \alpha : \lambda^\bot \),
  \[ \#\{ \beta : \lambda^\bot \mid \beta < \alpha \} \leq \#\{ \beta : \lambda^\bot \mid \beta \leq \alpha \} < \cof(\ord(\#\mu)) \]
\end{definition}
\begin{definition}[small]
  \label{def:Small}
  \uses{def:Params}
  A set \( s : \Set(\tau) \) is called \emph{small} if \( \#s < \#\kappa \).
  Smallness is stable under subset, intersection, small-indexed unions, symmetric difference, direct image, injective preimage, and many other operations (each of which should be its own lemma when formalised).
  Sets \( s, t : \Set(\tau) \) are called \emph{near} if \( s \symmdiff t \) is small; in this case, we write \( s \near t \).
  Nearness is an equivalence relation.
  If \( s \near t \) and \( u \) is small, then \( s \near (t \mathbin{\diamond} u) \), where \( \diamond \) is one of \( \cup, \cap, \setminus, \symmdiff \).
\end{definition}
\begin{definition}[litter]
  \label{def:Litter}
  \uses{def:Params}
  A \emph{litter} is a triple \( L = (\nu, \beta, \gamma) : \mu \times \lambda^\bot \times \lambda \) where \( \beta \neq \gamma \).
  The type of all litters is denoted \( \Litter \), and \( \#\Litter = \#\mu \).
\end{definition}
\begin{definition}[atom]
  \label{def:Atom}
  \uses{def:Litter}
  An \emph{atom} is a pair \( a = (L, i) : \Litter \times \kappa \).\footnote{This should be formalised as a structure, not as a definition. We should not use the projections of atoms unless absolutely necessary.}
  The type of all atoms is denoted \( \Atom \), and \( \#\Atom = \#\mu \).
  We write \( (-)^\circ : \Atom \to \Litter \) for the operation \( (L, i) \mapsto L \).\footnote{This must be a notation typeclass.}
  We write \( \LS(L) \coloneq \{ a \mid a^\circ = L \} \) for the \emph{litter set} of \( L \).\footnote{Maybe revise this name?}
\end{definition}
\begin{definition}[near-litter]
  \label{def:NearLitter}
  \uses{def:Litter,def:Atom}
  A \emph{near-litter} is a pair \( N = (L, s) : \Litter \times \Set \Atom \) such that \( s \near \LS(L) \).\footnote{Like with atoms, this should be a structure. We should create an actual constructor, rather than using the \( \langle - \rangle \) syntax.}
  We write \( (-)^\circ : \NearLitter \to \Litter \) for the operation \( (L, s) \mapsto L \).
  We write \( a \in N \) for \( a \in s \), where \( N = (L, s) \).
  Near-litters satisfy extensionality: there is at most one choice of \( L \) for each \( s \).
  Each near-litter has size \( \#\kappa \) when treated as a set of atoms.
  The type of all near-litters is denoted \( \NearLitter \), and \( \#\NearLitter = \#\mu \) (there are \( \#\mu \) litters, and \( \#\mu \) small sets of atoms by \cref{prop:mk_subset_mk_lt_cof}; the latter observation should be its own lemma).

  We have \( M \near N \) if and only if \( M^\circ = N^\circ \).
  If \( M^\circ = N^\circ \), then \( M \symmdiff N \) is small and \( M \cap N \) has size \( \#\kappa \).
  If \( M^\circ \neq N^\circ \), then \( M \cap N \) has size \( \#\kappa \) and \( M \cap N \) is small.
\end{definition}
\begin{definition}[base permutation]
  \label{def:BasePerm}
  \uses{def:Atom,def:Litter,def:NearLitter}
  A \emph{base permutation} is a pair \( \pi = (\pi^\Atom, \pi^\Litter) \), where \( \pi^\Atom \) is a permutation \( \Atom \simeq \Atom \) and \( \pi^\Litter \) is a permutation \( \Litter \simeq \Litter \), such that
  \[ \pi^\Atom[\LS(L)] \near \LS(\pi^\Litter(L)) \]
  Base permutations have a natural group structure, they act on atoms by \( \pi^\Atom \), they act on litters by \( \pi^\Litter \), and they act on near-litters by\footnote{We need to emphasise these properties, rather than emphasising the real definition \( \pi(N) = (\pi(N^\circ), \pi[N]) \).}
  \[ \pi(N)^\circ = \pi(N^\circ);\quad a \in \pi(N) \leftrightarrow a \in \pi[N] \]
  Base permutations are determined by their action on atoms.
  We should avoid directly referencing \( \pi^\Atom \) and \( \pi^\Litter \) whenever possible.
\end{definition}

\section{The structural hierarchy}

We will now establish the hierarchy of types that our model will be built inside.

\begin{definition}[path]
  \label{def:Path}
  \uses{def:TypeIndex}
  If \( \alpha, \beta \) are type indices, then a \emph{path} \( \alpha \tpath \beta \) is a nonempty finite subset of \( \lambda^\bot \) with maximal element \( \alpha \) and minimal element \( \beta \).\footnote{In previous versions, we used ordered lists (more precisely, a quiver structure) to represent paths. This had the effect that editing paths at the bottom was much easier than editing paths at the top. In this implementation, we are making the decision to put both types of edit on equal footing (and neither will definitionally reduce). TODO: It might be cleaner to still use a cons-list format but encapsulate it lots.}
  A path \( \alpha \tpath \bot \) is called an \emph{\( \alpha \)-extended index}.
  We write \( \nil(\alpha) \) for the path \( \{ \alpha \} : \alpha \tpath \alpha \).
  If \( h \) is a proof of \( \beta < \alpha \), we write \( \single(h) \) for the path \( \{ \alpha, \beta \} : \alpha \tpath \beta \).

  We have the inequality
  \begin{align*}
    \#(\alpha \tpath \beta)
    &\leq (\#\{ \gamma : \lambda^\bot \mid \gamma \leq \alpha \})^{<\omega} \\
    &= \max(\aleph_0, \#\{ \gamma : \lambda^\bot \mid \gamma \leq \alpha \}) \\
    &< \cof(\ord(\#\mu))
  \end{align*}
\end{definition}

Many of the objects in this construction have an associated type level \( \alpha : \lambda^\bot \), and by application of a path of the form \( \alpha \tpath \beta \) or \( \beta \tpath \alpha \), we can often define a new object of type level \( \beta \).
For this common task, we register the following notation typeclasses.
\begin{itemize}
  \item \( x \Downarrow A \) is the \emph{derivative} of an object of type \( \beta \) along a path \( A : \beta \tpath \gamma \), giving an object of type \( \gamma \);
  \item \( x \downarrow h \) abbreviates \( x \Downarrow \single(h) \);\footnote{In practice the typeclasses will probably not formally depend on each other, and this `abbreviation' may not be a definitional equality.}
  \item \( x \underset{\bot}{\Downarrow} A \) is the \emph{base derivative} of an object of type \( \beta \) along a path \( A : \beta \tpath \bot \);
  \item \( x \underset{\bot}{\downarrow} \) abbreviates \( x \underset{\bot}{\Downarrow} \single(h) \) where \( h \) is the canonical proof of \( \bot < \beta \) whenever \( \beta : \lambda \);
  \item \( x \Uparrow A \) is the \emph{coderivative} of an object of type \( \beta \) along a path \( A : \alpha \tpath \beta \), giving an object of type \( \alpha \);
  \item \( x \uparrow h \) abbreviates \( x \Uparrow \single(h) \).
\end{itemize}

When we say that an object has an associated type level in this context, we mean that the notation typeclass is registered in the following form.

\begin{verbatim}
class Derivative (X : Type _) (Y : outParam (Type _))
    (β : outParam TypeIndex) (γ : TypeIndex) where
  deriv : X → Path β γ → Y
\end{verbatim}

This means that when inferring the type of the expression \( x \Downarrow A \), we first compute the type of \( x \), which gives rise to a unique type index \( \beta \), then the type of \( A \) is inferred to give \( \gamma \), then the output type \( Y \) is uniquely determined.

The reason that we distinguish \( \underset{\bot}{\Downarrow} \) from \( \Downarrow \) is that the associated type \( Y \) is allowed to differ between the two forms.
We will give a brief example motivated by a definition we are about to make.
For each type index \( \beta \), there is a type of \( \beta \)-structural permutation, comprised of many base permutations.
If we have a path \( \beta \tpath \gamma \), we can convert a \( \beta \)-structural permutation into a \( \gamma \)-structural permutation; this will be the derivative map.
We will see that a given \( \bot \)-structural permutation contains exactly one base permutation, and so the types are in canonical isomorphism.
If \( x \) is a \( \beta \)-structural permutation and \( A : \beta \tpath \bot \), then \( x \Downarrow A \) is a \( \bot \)-structural permutation, and \( x \underset{\bot}{\Downarrow} A \) is the corresponding base permutation.

Because \( \uparrow, \downarrow \) and others are already used by Lean, we use slightly different notation in practice (e.g.\ \( \nearrow, \searrow \)).
In this writeup, however, we will use subscripts for derivatives and superscripts for coderivatives.
We will not distinguish typographically here between the single- and double-struck variants, or between \( \Downarrow \) and \( \underset{\bot}{\Downarrow} \); in the latter case, the syntax \( x_A \) always means \( x \underset{\bot}{\Downarrow} A \) whenever \( A \) has minimal element \( \bot \).
We will also use \( x_\gamma \) to denote the derivative \( x_h \) where \( h \) is some proof of \( \gamma < \beta \), and use \( x^\alpha \) to denote \( x^h \) where \( h : \beta < \alpha \).

\begin{definition-no-bp}[derivatives of paths]
  If \( A : \alpha \tpath \beta \) and \( B : \beta \tpath \gamma \), the derivative \( A_B \) is defined to be the union \( A \cup B : \alpha \tpath \gamma \), and the coderivative \( B^A \) is defined to be \( A_B \).
\end{definition-no-bp}

\begin{proposition}[induction along paths]
  \label{prop:Path.rec}
  \uses{def:Path}
  Fix a type index \( \alpha \) and universe \( u \), and let
  \[ C : \prod_{\beta : \lambda^\bot} (\alpha \tpath \beta) \to \Sort_u \]
  Let
  \[ n : C(\alpha, \nil(\alpha));\quad c : \prod_{\beta, \gamma : \lambda^\bot} \prod_{A : \alpha \tpath \beta} \prod_{h : \gamma < \beta} C(\beta, A) \to C(\beta, A_h) \]
  Then there is a term
  \[ x : \prod_{\beta : \lambda^\bot} \prod_{A : \alpha \tpath \beta} C(\beta, A) \]
  such that
  \[ x(\alpha, \nil(\alpha)) = n;\quad x(\gamma, A_h) = c(\beta, \gamma, A, h, x(\beta, A)) \]
  The same holds for paths pointing in the reverse direction.
\end{proposition}
\begin{definition}[tree]
  \label{def:Tree}
  \uses{def:Path}
  Let \( \tau \) be any type, and let \( \alpha \) be a type index.
  An \emph{\( \alpha \)-tree of \( \tau \)} is a function \( t \) that maps each \( \alpha \)-extended index \( A \) to an object \( t_A : \tau \); this defines its base derivatives.
  The type of \( \bot \)-trees of \( \tau \) is canonically isomorphic to \( \tau \).
  If \( t \) is an \( \alpha \)-tree and \( A : \alpha \tpath \beta \), we define the derivative \( t_A \) to be the \( \beta \)-tree defined by \( (t_A)_B = t_{(A_B)} \).
  This derivative map is functorial: for any paths \( A : \alpha \tpath \beta \) and \( B : \beta \tpath \gamma \), we have \( t_{(A_B)} = (t_A)_B \).
  If \( \tau \) has a group structure, then so does its type of trees: \( (t \cdot u)_A = t_A \cdot u_A \) and \( (t^{-1})_A = (t_A)^{-1} \).
  If \( \tau \) acts on \( \upsilon \), then \( \alpha \)-trees of \( \tau \) act on \( \alpha \)-trees of \( \upsilon \): \( (t(u))_A = t_A(u_A) \).

  If \( \#\tau \leq \#\mu \), there are at most \( \#\mu \)-many \( \alpha \)-trees of \( \tau \), since there are less than \( \cof(\ord(\#\mu)) \)-many \( \alpha \)-extended indices, allowing us to conclude by \cref{prop:mk_subset_mk_lt_cof} as each such tree is a subset of \( (\alpha \tpath \bot) \times \tau \) of size less than \( \cof(\ord(\#\mu)) \).
\end{definition}
\begin{definition}[structural permutation]
  \label{def:StrPerm}
  \uses{def:BasePerm,def:Tree}
  Let \( \alpha \) be a type index.
  Then an \emph{\( \alpha \)-structural permutation} (or just \emph{\( \alpha \)-permutation}) is an \( \alpha \)-tree of base permutations.
  The type of \( \alpha \)-permutations is written \( \StrPerm_\alpha \).
\end{definition}
As an implementation detail, we create a typeclass \( \texttt{StrPermClass}_\alpha \) for permutations that `act like' \( \alpha \)-permutations: they have a group structure and a canonical group embedding into \( \StrPerm_\alpha \).
When we quantify over structural permutations in this paper, it should be formalised using an additional quantification over \( \texttt{StrPermClass}_\alpha \).
\begin{definition}[enumeration]
  \label{def:Enumeration}
  \uses{def:Small}
  Let \( \tau \) be a type.
  An \emph{enumeration} of \( \tau \) is a pair \( E = (i, f) \) where \( i : \kappa \) and \( f \) is a partial function \( \kappa \to \tau \), such that all domain elements of \( f \) are strictly less than \( i \).\footnote{This should be encoded as a coinjective relation \( \kappa \to \tau \to \Prop \).}
  If \( x : \tau \), we write \( x \in E \) for \( x \in \ran f \).
  The set \( \{ y \mid y \in E \} \), which we may also loosely call \( E \), is small.
  We will write \( \varnothing \) for the empty enumeration \( (0, \varnothing) \).

  If \( g : \tau \to \sigma \), then \( g \) lifts to a direct image function mapping enumerations of \( \tau \) to enumerations of \( \sigma \):
  \[ g(i, f) = (i, f');\quad f' = \{ (j, g(x)) \mid (j, x) \in f \} \]
  Thus, \( x \in g(E) \leftrightarrow x \in g[E] \).
  In the same way, groups that act on \( \tau \) also act on enumerations of \( \tau \).\footnote{Actually, we should probably implement this using the inverse image not the direct image for better definitional equalities.}
  If \( g : \sigma \to \tau \) is injective, then \( g \) lifts to an inverse image function mapping enumerations of \( \tau \) to enumerations of \( \sigma \):
  \[ g^{-1}(i, f) = (i, f');\quad f' = \{ (j, x) \mid (j, g(x)) \in f \} \]
  This operation may cause the domain of \( f \) to shrink, but we will keep \( i \) the same.

  If \( E = (i, e) \) and \( F = (j, f) \) are enumerations of \( \tau \), we define their \emph{concatenation} by
  \[ E + F = (i + j, e \cup f');\quad f' = \{(i + k, x) \mid (k, x) \in f \} \]
  This operation commutes with the others: \( x \in E + F \leftrightarrow x \in E \vee x \in F \), \( g[E + F] = g[E] + g[F] \), and \( g^{-1}[E + F] = g^{-1}[E] + g^{-1}[F] \).

  We define a partial order on enumerations by setting \( (i, e) \leq (j, f) \) if and only if \( f \) extends \( e \) as a partial function.
  We obtain various corollaries, such as \( E \leq F \to g(E) \leq g(F) \) and \( E \leq E + F \).

  Every small subset of \( \tau \) is the range of some enumeration of \( \tau \).

  If \( \#\tau \leq \#\mu \), then there are at most \( \#\mu \)-many enumerations of \( \tau \): enumerations are determined by an element of \( \kappa \) and a small subset of \( \kappa \times \tau \).
  If \( \#\tau < \#\mu \), then there are strictly less than \( \#\mu \)-many enumerations of \( \tau \): use the same reasoning and apply \cref{prop:mk_subset_mk_lt_cof}.
\end{definition}
\begin{definition}[support]
  \label{def:StrSupport}
  \uses{def:Enumeration,def:Tree}
  A \emph{base support} is a pair \( S = (S^\Atom, S^\NearLitter) \) where \( S^\Atom \) is an enumeration of atoms and \( S^\NearLitter \) is an enumeration of near-litters.
  There are precisely \( \#\mu \) base supports.

  A \emph{\( \beta \)-structural support} (or just \emph{\( \beta \)-support}) is a \( \beta \)-tree of base supports.
  The type of \( \beta \)-supports is writte \( \StrSupp_\beta \).
  There are precisely \( \#\mu \) structural supports for any type index \( \beta \).
  Following \cref{def:Tree}, structural supports have derivatives, structural permutations act on structural supports, and there is a group action of \( \beta \)-permutations on \( \beta \)-supports.
  If \( S \) is a \( \beta \)-support and \( A : \alpha \tpath \beta \), the coderivative \( S^A \) is the \( \alpha \)-support given by
  \[ (S^A)_B = \begin{cases}
    S_C & \text{if } B = A_C \text{ for some } C \\
    \varnothing & \text{otherwise}
  \end{cases} \]
  Thus \( (S^A)_A = S \).

  Let \( \tau \) be a type, and let \( G \) be a \( \texttt{StrPermClass}_\beta \)-group that acts on \( \tau \).
  We say that \( S \) \emph{supports} \( x : \tau \) under the action of \( G \) if whenever \( \pi : G \) fixes \( S \), it also fixes \( x \).
\end{definition}
\begin{definition}[structural set]
  \label{def:StrSet}
  \uses{def:StrPerm}
  The type of \emph{\( \alpha \)-structural sets}, denoted \( \StrSet_\alpha \), is defined by well-founded recursion on \( \lambda^\bot \).
  \begin{itemize}
    \item \( \StrSet_\bot \coloneq \Atom \);
    \item \( \StrSet_\alpha \coloneq \prod_{\beta : \lambda^\bot} \beta < \alpha \to \StrSet_\beta \) where \( \alpha : \lambda \).
  \end{itemize}
  These equalities will in fact only be equivalences in the formalisation.
  We define the action of \( \alpha \)-permutations (more precisely, inhabitants of some type with a \( \texttt{StrPermClass}_\alpha \) instance) on \( \alpha \)-structural sets by well-founded recursion.
  \begin{itemize}
    \item \( \pi(x) = \pi_{\nil(\bot)}(x) \) if \( \alpha \equiv \bot \);
    \item \( \pi(x) = (\beta, h \mapsto \pi_h(x(\beta, h))) \) if \( \alpha : \lambda \).
  \end{itemize}
\end{definition}

\section{Position functions}
\begin{definition}[position function]
  \label{def:Position}
  \uses{def:Params}
  Let \( \tau \) be a type.
  A \emph{position function} for \( \tau \) is an injection \( \iota : \tau \to \mu \).
  This is a typeclass.
\end{definition}
\begin{proposition}[injective functions from denied sets]
  \label{prop:funOfDeny}
  \uses{def:Params}
  Let \( \tau \) be a type such that \( \#\tau \leq \#\mu \).
  Let \( D : \tau \to \Set(\mu) \) be a function such that for each \( x : \tau \), the set \( D(x) \), called the \emph{denied set} of \( x \), has size less than \( \cof(\ord(\#\mu)) \).
  Then there is an injective function \( f : \tau \to \mu \) such that if \( \nu \in D(x) \), then \( \nu < f(x) \).
\end{proposition}
\begin{proof}
  Pick a well-ordering \( \prec \) of \( \tau \) of length at most \( \ord(\#\mu) \).
  Define \( f \) by well-founded recursion on \( \prec \).
  Suppose that we have already defined \( f \) for all \( y \prec x \).
  Then let
  \[ X = \{ f(y) \mid y \prec x \} \cup D(x) \]
  This set has size less than \( \cof(\ord(\#\mu)) \), so it is not cofinal in \( \mu \).
  So there is some \( \nu : \mu \) such that \( \forall \nu' \in X,\, \nu > \nu' \).
  Set \( f(x) = \nu \).
\end{proof}
\begin{proposition}[base positions]
  \label{prop:BasePositions}
  \uses{def:Position}
  There are position functions on \( \Atom, \NearLitter \) that are jointly injective and satisfy
  \begin{itemize}
    \item \( \iota(\NL(a^\circ)) < \iota(a) \) for every atom \( a \);
    \item \( \iota(\NL(N^\circ)) \leq \iota(N) \) for every near-litter \( N \);
    \item \( \iota(a) < \iota(N) \) for every near-litter \( N \) and atom \( a \in N \symmdiff \LS(N^\circ) \).\footnote{TODO: Make syntax for \( N \symmdiff \LS(N^\circ) \).}
  \end{itemize}
  We register these position functions as instances for use in typeclass inference.
  We also define \( \iota(L) = \iota(\NL(L)) \) for litters.
\end{proposition}
\begin{proof}
  \uses{prop:funOfDeny}
  First, establish an equivalence \( f_\Litter : \Litter \simeq \mu \).
  Use \cref{prop:funOfDeny} to obtain an injective map \( f_\Atom : \NearLitter \to \mu \) such that \( f_\Litter(a^\circ) < f_\Atom(a) \) for each atom \( a \).
  Now use \cref{prop:funOfDeny} again to obtain an injective map \( f_\NearLitter : \NearLitter \to \mu \) such that
  \[ f_\Litter(N^\circ) < f_\NearLitter(N);\quad f_\Atom(a) < f_\NearLitter(N) \text{ for } a \in N \symmdiff \LS(N^\circ) \]
  Endow \( 3 \times \mu \) with its lexicographic order, of order type \( \ord(\#\mu) \), giving an order isomorphism \( e : 3 \times \mu \simeq \mu \).
  Finally, we define
  \[ \iota(a) = e(1, f_\Atom(a));\quad \iota(N) = \begin{cases}
    e(0, f_\Litter(N^\circ)) & \text{if } N = \NL(N^\circ) \\
    e(2, f_\NearLitter(N)) & \text{otherwise}
  \end{cases} \]
\end{proof}

\section{Hypotheses of the recursion}
\begin{definition}[model data]
  \label{def:ModelData}
  \uses{def:StrSupport,def:StrSet}
  Let \( \alpha \) be a type index.
  \emph{Model data} at type \( \alpha \) consists of:
  \begin{itemize}
    \item a \( \TSet_\alpha \) called the type of \emph{tangled sets} or \emph{t-sets}, which will be our type of model elements at level \( \alpha \), with an embedding \( U_\alpha : \TSet_\alpha \to \StrSet_\alpha \);
    \item a group \( \AllPerm_\alpha \) called the type of \emph{allowable permutations}, with a \( \texttt{StrPermClass}_\alpha \) instance and a specified action on \( \TSet_\alpha \),
  \end{itemize}
  such that
  \begin{itemize}
    \item if \( \pi : \AllPerm_\alpha \) and \( x : \TSet_\alpha \), then \( \pi(U_\alpha(x)) = U_\alpha(\pi(x)) \);
    \item every t-set of type \( \alpha \) has an \( \alpha \)-support (\cref{def:StrSupport}) for its action under the \( \alpha \)-allowable permutations;
  \end{itemize}
\end{definition}
\begin{definition}[tangle]
  \label{def:Tangle}
  \uses{def:ModelData}
  Let \( \alpha \) be a type index with model data.
  The type \( \Tang_\alpha \) of \emph{\( \alpha \)-tangles}, as well as the operations \( \set : \Tang_\alpha \to \TSet_\alpha \) and \( \supp : \Tang_\alpha \to \StrSupp_\alpha \) are defined by cases.
  \begin{itemize}
    \item If \( \alpha : \lambda \), an \( \alpha \)-tangle is a pair \( t = (x, S) \) where \( x \) is a tangled set of type \( \alpha \) and \( S \) is a support for \( x \).
    We define \( \set(t) = x \) and \( \supp(t) = S \).
    \item A \( \bot \)-tangle is a t-set \( x : \TSet_\bot \).\footnote{This is not the same as an atom: given arbitrary model data at type \( \bot \), we have no way of knowing if its \( \TSet \) is the same as \( \Atom \), all that we know is that \( \TSet_\bot \) embeds into \( \Atom \).}
    We define \( \set(x) = x \), and make \( \supp(x) \) be the \( \bot \)-support corresponding to the base support \( (1, f) \) where \( (0, a) \in f \) and \( a \) is the atom corresponding to the type-\( \bot \) t-set \( U_\bot(x)\).
  \end{itemize}
  Allowable permutations \( \pi \) act on tangles \( t \) in the following way.
  \begin{itemize}
    \item If \( \alpha : \lambda \), then \( \pi((x, S)) = (\pi(x), \pi(S)) \).
    \item If \( \alpha \equiv \bot \), then \( \pi(x) \) is already defined.
  \end{itemize}
  In each case, \( \supp(t) \) supports \( t \) for its action under the allowable permutations.
\end{definition}
\begin{definition}[typed near-litter]
  \label{def:TypedNearLitter}
  \uses{def:ModelData,def:Tangle,prop:BasePositions}
  Let \( \alpha \) be a proper type index with model data, and suppose that \( \Tang_\alpha \) has a position function.
  We say that \( \alpha \) has \emph{typed near-litters} if there is a function \( \typed_\alpha : \NearLitter \to \TSet_\alpha \) such that
  \begin{itemize}
    \item if \( \pi : \AllPerm_\alpha \) and \( N : \NearLitter \), then \( \pi(\typed_\alpha(N)) = \typed_\alpha(\pi_\bot(N)) \); and
    \item if \( N \) is a near-litter and \( t \) is an \( \alpha \)-tangle such that \( \set(t) = \typed_\alpha(N) \), we have \( \iota(N) \leq \iota(t) \).
  \end{itemize}
  Objects of the form \( \typed_\alpha \) are called \emph{typed near-litters}.
\end{definition}
\begin{definition}[coherent data]
  \label{def:CoherentData}
  \uses{def:TypedNearLitter,prop:Path.rec}
  Let \( \alpha \) be a proper type index.
  Suppose that we have model data for all type indices \( \beta \leq \alpha \), position functions for \( \Tang_\beta \) for all \( \beta < \alpha \), and typed near-litters for all \( \beta < \alpha \).
  We say that the model data is \emph{coherent} below level \( \alpha \) if the following hold.
  \begin{itemize}
    \item For \( \gamma < \beta \leq \alpha \), there is a map \( \AllPerm_\beta \to \AllPerm_\gamma \) that commutes with the coercions from \( \AllPerm_{(-)} \) to \( \StrPerm_{(-)} \).
    We can use this map to define arbitrary derivatives of allowable permutations by \cref{prop:Path.rec}.
    \item If \( (x, S) \) is a \( \beta \)-tangle for \( \beta < \alpha \), and \( y \) is an atom or near-litter\footnote{We might want to encapsulate atoms and near-litters somehow. We could make a typeclass, or write theorems in terms of the coproduct \( \Atom \oplus \NearLitter \).} that occurs in the range of \( S_A \), then \( \iota(y) < \iota(x, S) \).
    \item TODO: An \( f \)-map condition.
  \end{itemize}
\end{definition}
