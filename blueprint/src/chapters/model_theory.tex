In this chapter, we establish the model theory that is used to derive the consistency of \( \NF \) from that of \( \TTT \).
This requires much of mathlib's model theory library to be completely rewritten for the many-sorted case, which is (so far) not an objective of this project.
Some potential design decisions are considered in the following sections.
On the whole, this chapter should be considered just a sketch of the main argument: sufficient for a human reader, but insufficient for a detailed blueprint.

\section{Many-sorted model theory}
This is loosely based off the perspective on categorical logic offered by Johnstone in Volume 2 of \emph{Sketches of an Elephant}, and takes heavy inspiration from the \emph{Flypitch project}.
\begin{definition}
  A \emph{\( \Sigma \)-language} consists of a map \( \newoperator{Functions} : \prod_{n : \mathbb N} (\Fin n \to \Sigma) \to \Sigma \to \Type_u \) and a map \( \newoperator{Relations} : \prod_{n : \mathbb N} (\Fin n \to \Sigma) \to \Type_v \).
\end{definition}
\begin{definition}
  Let \( \Phi : \Sigma \to \Sigma' \).
  Let \( L \) be a \( \Sigma \)-language and let \( L' \) be a \( \Sigma' \)-language.
  Then a \emph{\( \Phi \)-morphism} of languages \( L \xrightarrow\Phi L' \) consists of a map
  \[ \newoperator{onFunction} : \prod_{n : \mathbb N} \prod_{A : \Fin n \to \Sigma} \prod_{B : \Sigma} \newoperator{Functions}_L(n, A, B) \to \newoperator{Functions}_{L'}(\Phi \circ A, \Phi(B)) \]
  and a map
  \[ \newoperator{onRelation} : \prod_{n : \mathbb N} \prod_{A : \Fin n \to \Sigma} \newoperator{Relations}_L(n, A) \to \newoperator{Functions}_{L'}(\Phi \circ A) \]
  If \( L, L' \) are \( \Sigma \)-languages, we may simply say that a \emph{morphism} of languages \( L \to L' \) is a \( \id_\Sigma \)-morphism \( L \xrightarrow{\id_\Sigma} L' \).\footnote{It is crucial that \( \id \circ f \equiv f \circ \id \equiv f \) definitionally.}
\end{definition}
\begin{definition}
  Let \( L, L' \) be \( \Sigma \)-languages.
  We define \( L \oplus L' \) to be the \( \Sigma \)-language with
  \[ \newoperator{Functions}(n, A, B) = \newoperator{Functions}_L(n, A, B) \oplus \newoperator{Functions}_{L'}(n, A, B) \]
  and
  \[ \newoperator{Relations}(n, A) = \newoperator{Relations}_L(n, A) \oplus \newoperator{Relations}_{L'}(n, A) \]
  There are morphisms of languages \( L, L' \to L \oplus L' \).
\end{definition}
\begin{definition}
  Let \( L \) be a \( \Sigma \)-language, and let \( M : \Sigma \to \Type_w \).
  An \emph{\( L \)-structure} on \( M \) consists of a map
  \[ \newoperator{funMap} : \prod_{n : \mathbb N} \prod_{A : \Fin n \to \Sigma} \prod_{B : \Sigma} \newoperator{Functions}(n, A, B) \to \left(\prod_{i : \Fin n} M(A(i)) \right) \to M(B) \]
  and a map
  \[ \newoperator{relMap} : \prod_{n : \mathbb N} \prod_{A : \Fin n \to \Sigma} \newoperator{Relations}(n, A) \to \left(\prod_{i : \Fin n} M(A(i)) \right) \to \Prop \]
  We will write \( f^M \) for \( \newoperator{funMap}(n, A, B, f) \), and similarly \( R^M \) for \( \newoperator{relMap}(n, R) \).
  If \( M \) is an \( L \)-structure and an \( L' \)-structure, it is also naturally an \( (L \oplus L') \)-structure.
  If \( M \) is an \( L \)-structure and an \( L' \)-structure, then a morphism of \( \Sigma \)-languages \( \Phi : L \to L' \) is called an \emph{expansion} on \( M \) if it commutes with the interpretations of all symbols on \( M \).
\end{definition}
\begin{definition}
  Let \( L \) be a \( \Sigma \)-language.
  A \emph{morphism} of \( L \)-structures \( M \to N \) consists of functions \( h_A : M_A \to N_A \) such that for all \( n : \mathbb N \), \( A : \Fin n \to \Sigma \), and \( B : \Sigma \), all function symbols \( f : \newoperator{Functions}(n, A, B) \), and all \( x : \prod_{i : \Fin n} M(A(i)) \),
  \[ h_B(f^M(x)) = f^N(i \mapsto h_{A(i)}(x(i))) \]
  and for all relation symbols \( R : \newoperator{Relations}(n, A) \),
  \[ R^M(x) \to R^N(i \mapsto h_{A(i)}(x(i))) \]
\end{definition}
\begin{definition}
  Let \( L \) be a \( \Sigma \)-language, and let \( \alpha : \Type_{u'} \) be a sort of variables of sort \( S : \alpha \to \Sigma \).
  An \emph{\( L \)-term on \( \alpha : S \) of sort \( A \)}, the type of which is denoted \( \Term_{\alpha:S} A \), is either
  \begin{itemize}
    \item a variable, comprised solely of a name \( n : \alpha \) such that \( S(n) = A \), or
    \item an application of a function symbol, comprised of some \( n : \mathbb N \), a map \( B : \Fin n \to \Sigma \), a function symbol \( f : \newoperator{Functions}(n, B, A) \), and terms \( t : \prod_{i : \Fin n} \Term_{\alpha:S} B(i) \).
  \end{itemize}
\end{definition}
\begin{definition}
  Let \( L \) be a \( \Sigma \)-language, and let \( \alpha : \Type_{u'} \) and \( S : \alpha \to \Sigma \).
  We define the type of \emph{\( L \)-bounded formulae on \( \alpha : S \)} with free variables indexed by \( \alpha \), and \( n \) additional free variables of sorts \( f : \Fin n \to \Sigma \), denoted \( \BForm_{\alpha:S}^f \), by the following constructors.
  \begin{itemize}
    \item \( \newoperator{falsum} : \BForm_{\alpha:S}^f \);
    \item \( \newoperator{equal} : \prod_{A : \Sigma} \Term_{\alpha \oplus \Fin n : S \oplus f} A \to \Term_{\alpha \oplus \Fin n : S \oplus f} A \to \BForm_{\alpha:S}^f \);
    \item \( \newoperator{rel} : \prod_{m : \mathbb N} \prod_{B : \Fin m \to \Sigma} \prod_{R : \newoperator{Relations}(m, B)} \left( \prod_{i : \Fin m} \Term_{\alpha \oplus \Fin n : S \oplus f} B(i) \right) \to \BForm_{\alpha:S}^f \);
    \item \( \newoperator{imp} : \BForm_{\alpha:S}^f \to \BForm_{\alpha:S}^f \to \BForm_{\alpha:S}^f \); and
    \item \( \newoperator{all} : \prod_{A : \Sigma} \BForm_{\alpha:S}^{A :: f} \to \BForm_{\alpha:S}^f \),
  \end{itemize}
  where the syntax \( A :: f \) denotes the cons operation on maps from \( \Fin n \).\footnote{This is implemented in mathlib as \texttt{Fin.cons}.}
  An \emph{\( L \)-formula on \( \alpha : S \)} with free variables indexed by \( \alpha \) is an inhabitant of \( \BForm_{\alpha:S}^{\#[]} \).\footnote{The expression \( \#[] \) is mathlib's syntax for the unique map \( \Fin 0 \to \Sigma \).}
  An \emph{\( L \)-sentence} is an \( L \)-formula with free variables indexed by \( \newoperator{Empty} : f \), where \( f \) is the unique function \( \newoperator{Empty} \to \Sigma \).
  An \emph{\( L \)-theory} is a set of \( L \)-sentences.
\end{definition}

\section{Term models}
% <https://github.com/flypitch/flypitch/blob/aea5800db1f4cce53fc4a113711454b27388ecf8/src/henkin.lean>
\begin{definition}
  Let \( L \) be a \( \Sigma \)-language.
  A \emph{1-\( L \)-formula} of sort \( A \) is an \( L \)-bounded formula on \( \newoperator{Empty} \) with one additional free variable of sort \( f : \Fin 1 \to \Sigma \) given by \( f(x) = A \).
\end{definition}
\begin{definition}
  Let \( L \) be a \( \Sigma \)-language.
  The \emph{witness symbols} for \( L \) is the language \( L_W \) consisting of a single constant of sort \( A \) for every 1-\( L \)-formula of sort \( A \).
\end{definition}
\begin{proposition}
  Let \( M \) be a nonempty \( L \)-structure.
  Then \( M \) has an \( L_W \)-structure such that for each 1-\( L \)-formula \( \phi \) of sort \( A \), there is a constant symbol \( c : \newoperator{Functions}(0, \#[], A) \) such that
  \[ \forall x : M_A,\, \phi(x) \to \phi(c) \]
\end{proposition}
\begin{proof}
  We define the interpretation of the constant for \( \phi \) to be some \( x : M \) such that \( M \vDash \phi(x) \) if one exists, or an arbitrary \( y : M \) otherwise.\footnote{Use mathlib's \texttt{Classical.epsilon}.}
  This defines an \( L_W \)-structure for \( M \), and clearly \( M \) satisfies the required property.
\end{proof}
\begin{definition}
  For each \( n : \mathbb N \), we define
  \[ L^{(0)} = L;\quad L^{(n+1)} = L^{(n)} \oplus (L^{(n)})_W \]
  This forms a directed diagram of languages, which has a colimit \( L^{(\omega)} \).
  There are natural morphisms \( L \to L^{(\omega)} \) and \( L^{(n)} \to L^{(\omega)} \) which are expansions on \( M \).
\end{definition}
\begin{proposition}
  Let \( M \) be a nonempty \( L \)-structure.
  Then \( M \) has an \( L^{(\omega)} \)-structure such that for each 1-\( L^{(\omega)} \)-formula \( \phi \) of sort \( A \), there is a constant symbol \( c : \newoperator{Functions}(0, \#[], A) \) such that
  \[ \forall x : M_A,\, \phi(x) \to \phi(c) \]
  Then, by the Tarski--Vaught test (which must be proven in the many-sorted case), the set \( N \subseteq M \) comprised of all of the interpretations of \( L^{(\omega)} \)-terms, is the domain of an \( L^{(\omega)} \)-elementary substructure of \( M \).
\end{proposition}
This will take substantial work.

\section{Ambiguity}
\begin{definition}
  Let \( L \) be an \( \mathbb N \)-language.
  A \emph{type raising morphism} is a map of languages \( L \xrightarrow{\newoperator{succ}} L \), where \( \newoperator{succ} : \mathbb N \to \mathbb N \) is the successor function.
\end{definition}
TODO
