\section{Relations}

\begin{definition-no-bp}
  \label{def:coinj_cosurj}
  Let \( R : \sigma \to \tau \to Prop \).
  We say that \( R \) is
  \begin{itemize}
    \item \emph{injective}, if \( s_1 \mathrel{R} t, s_2 \mathrel{R} t \) imply \( s_1 = s_2 \);
    \item \emph{surjective}, if for every \( t : \tau \), there is some \( s : \sigma \) such that \( s \mathrel{R} t \);
    \item \emph{coinjective}, if \( s \mathrel{R} t_1, s \mathrel{R} t_2 \) imply \( t_1 = t_2 \);
    \item \emph{cosurjective}, if for every \( s : \sigma \), there is some \( t : \tau \) such that \( s \mathrel{R} t \);
    \item \emph{functional}, if \( R \) is coinjective and cosurjective, or equivalently, for every \( s : \sigma \) there is exactly one \( t : \tau \) such that \( s \mathrel{R} t \);
    \item \emph{cofunctional}, if \( R \) is injective and surjective, or equivalently, for every \( t : \tau \) there is exactly one \( s : \sigma \) such that \( s \mathrel{R} t \);
    \item \emph{one-to-one}, if \( R \) is injective and coinjective;
    \item \emph{bijective}, if \( R \) is functional and cofunctional.
  \end{itemize}
  These definitions are from \url{https://www.kylem.net/math/relations.html}, and most of these are in mathlib under \texttt{Mathlib.Logic.Relator}.
\end{definition-no-bp}

\section{Cardinal arithmetic}

\begin{lemma-no-bp}[mathlib]
  \label{prop:mk_subset_mk_lt_cof}
  Let \( \#\mu \) be a strong limit cardinal.
  Then there are precisely \( \#\mu \)-many subsets of \( \mu \) of size strictly less than \( \cof(\ord(\#\mu)) \).
\end{lemma-no-bp}
\begin{proof}
  Endow \( \mu \) with its initial well-ordering.
  Each such subset is bounded in \( \mu \) with respect to this well-ordering as its size is less than \( \cof(\ord(\#\mu)) \).
  So it suffices to prove there are precisely \( \#\mu \)-many bounded subsets of \( \mu \).
  \begin{align*}
    \#\{ s : \Set \mu \mid \exists \nu : \mu,\, \forall x \in s,\, x < \nu \}
    &= \#\bigcup_{\nu : \mu} \{ s : \Set \mu \mid \forall x \in s,\, x < \nu \} \\
    &\leq \sum_{\nu : \mu} \#\{ s : \Set \mu \mid \forall x \in s,\, x < \nu \} \\
    &= \sum_{\nu : \mu} \#\{ s : \Set \{ x : \mu \mid x < \nu \}\} \\
    &= \sum_{\nu : \mu} \underbrace{2^{\#\{ x : \mu \mid x < \nu \}}}_{<\mu} \\
    &\leq \mu
  \end{align*}
\end{proof}
