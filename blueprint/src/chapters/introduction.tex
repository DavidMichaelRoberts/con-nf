\section{Overview}
\label{s:overview}

We will begin by discussing the mathematical background for the question of the consistency of \NF.
We will establish the mathematical context for the proof we will present.
In particular, our proof will not directly show the consistency of \NF; instead, we will construct a model of a related theory known as \emph{tangled type theory}, or \TTT.
We will show that there is a structure that satisfies a particular axiomatisation of {\TTT} which we will discuss in \cref{ss:theories:ttt}.
The expected conclusion that {\NF} is consistent then follows from the fact that {\NF} and {\TTT} are equiconsistent \cite{holmes-ttt}.

We will now outline our general strategy for the construction of a model of tangled type theory.
As we will outline in \cref{ss:theories:ttt}, {\TTT} is a many-sorted theory with types indexed by a limit ordinal \( \lambda \).
In order to impose symmetry conditions on our structure, we will add an additional level of objects below type zero.
These will not be a part of the final model we construct.
This base type will be comprised of objects called \emph{atoms} (although they are not atoms in the traditional model-theoretic sense).
Alongside the construction of the types of our model, we will also construct a group of permutations of each type, called the \emph{allowable permutations}.
Such permutations will preserve the structure of the model in a strong sense; for instance, they preserve membership.

The construction of a given type can only be done under certain hypotheses on the construction of lower types.
The most restrictive condition that we will need to satisfy is a bound on the size of each type.
In order to do this, we will need to show that there are a lot of allowable permutations.
The main technical lemma establishing this, called the \emph{freedom of action theorem}, roughly states that a partial function that locally behaves like an allowable permutation can be extended to an allowable permutation.
It, and its various corollaries, will be outlined in more detail when we are in a position to prove it.

We can then finish the main induction to build the entire model out of the types we have constructed.
This step, while invisible to a human reader in set theory, takes substantial effort to formally establish in a dependent type theory.
It then remains to show that this is a model of {\TTT} as desired, or alternatively, a model of a particular finite axiomatisation.

Finally, we will set out to formally prove the consistency of \NF.
This will involve breaking down the model-theoretic arguments from \cite{typical-ambiguity,holmes-ttt} into concrete formalisable lemmas.

Implementation details will be discussed in footnotes.

\section{The simple theory of types}

In 1937, Quine introduced \emph{New Foundations} (\NF) \cite{quine-nf}, a set theory with a very small collection of axioms.
To give a proper exposition of the theory that we intend to prove consistent, we will first make a digression to introduce the related theory {\TST}, as explained in \cite{con-nf}.

The \emph{simple theory of types} (known as \emph{théorie simple des types} or {\TST}) is a first order set theory with several sorts, indexed by the nonnegative integers.
Each sort, called a \emph{type}, is comprised of \emph{sets} of that type; each variable \( \mquote{x} \) has a nonnegative integer \( \ttype(\mquote x) \) which denotes the type it belongs to.
For convenience, we may write \( x^n \) to denote a variable \( x \) with type \( n \).

The primitive predicates of this theory are equality and membership.
An equality \( \mquote{x = y} \) is a well-formed formula precisely when \( \ttype(\mquote{x}) = \ttype(\mquote{y}) \), and similarly a membership formula \( \mquote{x \in y} \) is well-formed precisely when \( \ttype(\mquote{x}) + 1 = \ttype(\mquote{y}) \).

The axioms of this theory are extensionality
\[ \forall x^{n + 1},\, \forall y^{n + 1},\, (\forall z^n,\, z^n \in x^{n+1} \leftrightarrow z^n \in y^{n+1}) \to x^{n+1} = y^{n+1} \]
and comprehension
\[ \exists x^{n + 1},\, \forall y^n,\, (y^n \in x^{n+1} \leftrightarrow \varphi(y^n)) \]
where \( \varphi \) is any well-formed formula, possibly with parameters.

Note that these are both axiom schemes, with instances for all type levels \( n \), and (in the latter case) for all well-formed formulae \( \varphi \).
Because extensionality at level \( n + 1 \) requires us to talk about sets at level \( n \), the inhabitants of type 0, called \emph{individuals}, cannot be examined using these axioms.

By comprehension, there is a set at each nonzero type that contains all sets of the previous type.
However, Russell-style paradoxes are avoided as formulae of the form \( x^n \in x^n \) are ill-formed.

\section{New Foundations}

New Foundations is a one-sorted first-order theory based on {\TST}.
Its primitive propositions are equality and membership.
There are no well-formedness constraints on these primitive propositions.

Its axioms are precisely the axioms of {\TST} with all type annotations erased.
That is, it has an axiom of extensionality
\[ \forall x,\, \forall y,\, (\forall z,\, z \in x \leftrightarrow z \in y) \to x = y \]
and an axiom scheme of comprehension
\[ \exists x,\, \forall y,\, (y \in x \leftrightarrow \varphi(y)) \]
the latter of which is defined for those formulae \( \varphi \) that can be obtained by erasing the type annotations of a well-formed formula of {\TST}.
Such formulae are called \emph{stratified}.
To avoid the explicit dependence on {\TST}, we can equivalently characterise the stratified formulae as follows.
A formula \( \varphi \) is said to be stratified when there is a function \( \sigma \) from the set of variables to the nonnegative integers, in such a way that for each subformula \( \mquote{x = y} \) of \( \varphi \) we have \( \sigma(\mquote{x}) = \sigma(\mquote{y}) \), and for each subformula \( \mquote{x \in y} \) we have \( \sigma(\mquote{x}) + 1 = \sigma(\mquote{y}) \).

It is important to emphasise that while the axioms come from a many-sorted theory, {\NF} is not one; it is well-formed to ask if any set is a member of, or equal to, any other.

Russell's paradox is avoided because the set \( \{ x \mid x \notin x \} \) cannot be formed; indeed, \( x \notin x \) is an unstratified formula.
Note, however, that the set \( \{ x \mid x = x \} \) is well-formed, and so we have a universal set \( V \).
Specker showed in \cite{specker-choice-nf} that {\NF} disproves the Axiom of Choice, and Hailperin showed in \cite{hailperin-finite-axiomatisation} that {\NF} is finitely axiomatisable.

While our main result is that New Foundations is consistent, we attack the problem by means of an indirection through a third theory.

\section{Tangled type theory}
\label{ss:theories:ttt}

Introduced by Holmes in \cite{holmes-ttt}, \emph{tangled type theory} ({\TTT}) is a multi-sorted first order theory based on {\TST}.
This theory is parametrised by a limit ordinal \( \lambda \), the elements of which will index the sorts; \( \omega \) works, but we prefer generality.
As in {\TST}, each variable \( \mquote{x} \) has a type that it belongs to, denoted \( \ttype(\mquote x) \).
However, in {\TTT}, this is not a positive integer, but an element of \( \lambda \).

The primitive predicates of this theory are equality and membership.
An equality \( \mquote{x = y} \) is a well-formed formula when \( \ttype(\mquote{x}) = \ttype(\mquote{y}) \).
A membership formula \( \mquote{x \in y} \) is well-formed when \( \ttype(\mquote{x}) < \ttype(\mquote{y}) \).

The axioms of {\TTT} are obtained by taking the axioms of {\TST} and replacing all type indices in a consistent way with elements of \( \lambda \).
More precisely, for any order-embedding \( s : \omega \to \lambda \), we can convert a well-formed formula \( \varphi \) of {\TST} into a well-formed formula \( \varphi^s \) of {\TTT} by replacing each type variable \( \alpha \) with \( s(\alpha) \).

It is important to note that membership across types in {\TTT} behaves in some quite bizarre ways.
Let \( \alpha \in \lambda \), and let \( x \) be a set of type \( \alpha \).
For any \( \beta < \alpha \), the extensionality axiom implies that \( x \) is uniquely determined by its type-\( \beta \) elements.
However, it is simultaneously determined by its type-\( \gamma \) elements for any \( \gamma < \alpha \).
In this way, one extension of a set controls all of the other extensions.

The comprehension axiom allows a set to be built which has a specified extension in a single type.
The elements not of this type may be considered `controlled junk'.

We now present the following striking theorem, which we will prove a version of in detail in \cref{c:model_theory}.

\begin{theorem*}[Holmes]
  {\NF} is consistent if and only if {\TTT} is consistent. \cite{holmes-ttt}
\end{theorem*}

Thus, our task of proving {\NF} consistent is reduced to the task of proving {\TTT} consistent.
We will do this by exhibiting an explicit model (albeit one that requires a great deal of Choice to construct).
As {\TTT} has types indexed by a limit ordinal, and sets can only contain sets of lower type, we can construct a model by recursion over \( \lambda \).
In particular, a model of {\TTT} is a well-founded structure.
This was not an option with {\NF} directly, as the universal set \( V = \{ x \mid x = x \} \) would necessarily be constructed before many of its elements.

\section{Finitely axiomatising tangled type theory}

As mentioned above, Hailperin showed in \cite{hailperin-finite-axiomatisation} that the comprehension scheme of {\NF} is equivalent to a finite conjunction of its instances.
In fact, the same proof shows that the comprehension scheme of {\TST} (and hence that of {\TTT}) is equivalent to a finite conjunction of its instances, but instantiated at all possible type sequences.

We will exhibit one such collection of instances here, totalling eleven axioms.
Our choice is inspired by those used in the Metamath implementation of Hailperin's algorithm in \cite{metamath-nf}.
In the following table, the notation \( \langle a, b \rangle \) denotes the Kuratowski pair \( \{ \{ a \}, \{ a, b \} \} \).
The first column is Hailperin's name for the axiom, and the second is (usually) the name from \cite{metamath-nf}.
\begin{center}
  \begin{tabular}{llcl}
    % \( - \) & extensionality & \( \forall x^1,\, \forall y^1,\, (\forall z^0,\, z \in x \leftrightarrow z \in y) \to x = y \) \\
    P1(a) & intersection & \( \forall x^1 y^1,\, \exists z^1,\, \forall w^0,\, w \in z \leftrightarrow (w \in x \wedge w \in y) \) \\
    P1(b) & complement & \( \forall x^1,\, \exists z^1,\, \forall w^0,\, w \in z \leftrightarrow w \notin x \) \\
    \( - \) & singleton & \( \forall x^0,\, \exists y^1,\, \forall z^0,\, z \in y \leftrightarrow z = x \) \\
    P2 & singleton image & \( \forall x^3,\, \exists y^4,\, \forall z^0 w^0,\, \langle \{ z \}, \{ w \} \rangle \in y \leftrightarrow \langle z, w \rangle \in x \) \\
    P3 & insertion two & \( \forall x^3,\, \exists y^5,\, \forall z^0 w^0 t^0,\, \langle \{ \{ z \} \}, \langle w, t \rangle \rangle \in y \leftrightarrow \langle z, t \rangle \in x \) \\
    P4 & insertion three & \( \forall x^3,\, \exists y^5,\, \forall z^0 w^0 t^0,\, \langle \{ \{ z \} \}, \langle w, t \rangle \rangle \in y \leftrightarrow \langle z, w \rangle \in x \) \\
    P5 & cross product & \( \forall x^1,\, \exists y^3,\, \forall z^2,\, z \in y \leftrightarrow \exists w^0 t^0,\, z = \langle w, t \rangle \wedge t \in x \) \\
    P6 & type lowering & \( \forall x^4,\, \exists y^1,\, \forall z^0,\, z \in y \leftrightarrow \forall w^1,\, \langle w, \{ z \} \rangle \in x \) \\
    P7 & converse & \( \forall x^3,\, \exists y^3,\, \forall z^0 w^0,\, \langle z, w \rangle \in y \leftrightarrow \langle w, z \rangle \in x \) \\
    P8 & cardinal one & \( \exists x^2,\, \forall y^1,\, y \in x \leftrightarrow \exists z^0,\, \forall w,\, w \in y \leftrightarrow w = z \) \\
    P9 & subset & \( \exists x^4,\, \forall y^1 z^1,\, \langle y, z \rangle \in x \leftrightarrow \forall w^0,\, w \in y \to w \in z \)
  \end{tabular}
\end{center}
Axioms P1--P9 except for P6 are \emph{predicative}: they stipulate the existence of a set with type at least that of all of the parameters.
It is relatively straightforward to prove the consistency of predicative {\TTT}, and we will see later that the proof of P6 in our model takes a different form to the proofs of the other axioms.
