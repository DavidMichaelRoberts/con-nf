\theoremstyle{definition}

\newtheorem*{theorem*}{Theorem}

\newtheorem{theorem}{Theorem}[chapter]
\newtheorem{proposition}[theorem]{Proposition}
\newtheorem{lemma}[theorem]{Lemma}
\newtheorem{corollary}[theorem]{Corollary}
\newtheorem{definition}[theorem]{Definition}

% Environments that don't add nodes to the blueprint.
\newtheorem{theorem-no-bp}[theorem]{Theorem}
\newtheorem{proposition-no-bp}[theorem]{Proposition}
\newtheorem{lemma-no-bp}[theorem]{Lemma}
\newtheorem{corollary-no-bp}[theorem]{Corollary}
\newtheorem{definition-no-bp}[theorem]{Definition}

\theoremstyle{remark}
\newtheorem{remark}[theorem]{Remark}
\newtheorem{remarks}[theorem]{Remarks}
\newtheorem{example}[theorem]{Example}
\newtheorem{examples}[theorem]{Examples}

\setoperatorfont\mathsf

% This is a workaround to make PlasTeX and XeLaTeX do the same thing: use the mathsf font for new operators.
\newcommand{\newoperator}[1]{\operatorname{\mathsf{#1}}}

\newcommand{\Con}{\newoperator{Con}}
\newcommand{\ttype}{\newoperator{type}}
\newcommand{\mquote}[1]{\ensuremath{\text{‘}#1\text{’}}}
\newcommand{\ot}{\newoperator{ot}}
\newcommand{\ord}{\newoperator{ord}}
\newcommand{\cof}{\newoperator{cof}}
\newcommand{\dom}{\newoperator{dom}}
\newcommand{\ran}{\newoperator{ran}}
\newcommand{\Set}{\newoperator{Set}}

\newcommand{\TSet}{\mathsf{TSet}}
\newcommand{\Tang}{\mathsf{Tang}}
\newcommand{\StrSet}{\mathsf{StrSet}}
\newcommand{\StrSupp}{\mathsf{StrSupp}}
\newcommand{\AllPerm}{\mathsf{AllPerm}}
\newcommand{\StrPerm}{\mathsf{StrPerm}}

\newcommand{\NF}{\ensuremath{\mathsf{NF}}}
\newcommand{\TTT}{\ensuremath{\mathsf{TTT}}}
\newcommand{\TST}{\ensuremath{\mathsf{TST}}}

\newcommand{\Ord}{\mathsf{Ord}}
\newcommand{\Card}{\mathsf{Card}}
\newcommand{\Prop}{\mathsf{Prop}}
\newcommand{\Type}{\mathsf{Type}}
\newcommand{\Sort}{\mathsf{Sort}}

\newcommand{\Atom}{{\mathcal A}}
\newcommand{\Litter}{{\mathcal L}}
\newcommand{\NearLitter}{{\mathcal N}}
\newcommand{\NL}{\newoperator{NL}}
\newcommand{\LS}{\newoperator{LS}}

\newcommand{\near}{\mathrel{\overset{N}{\sim}}}
\newcommand{\tpath}{\rightsquigarrow} % "type path"
\newcommand{\single}{\newoperator{single}}
\newcommand{\nil}{\newoperator{nil}}

\newcommand{\typed}{\newoperator{typed}}
\newcommand{\set}{\newoperator{set}}
\newcommand{\supp}{\newoperator{supp}}
