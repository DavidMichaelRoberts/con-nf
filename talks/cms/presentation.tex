\documentclass[xcolor=dvipsnames]{beamer}

\usepackage[UKenglish]{babel}

\usefonttheme{professionalfonts}
\usepackage{fontspec}
\setromanfont{TeX Gyre Pagella}
\setsansfont{Lato}

\usepackage{mathtools}

\usepackage{unicode-math}
\setmathfont{TeX Gyre Pagella Math}

\usepackage{physics}

\usepackage{pgfpages}
%\setbeameroption{show notes on second screen=right}

\definecolor{UBCblue}{rgb}{0.04706, 0.13725, 0.26667}
\definecolor{UBCgreen}{rgb}{0.04706, 0.26667, 0.13725}

\usecolortheme[named=UBCblue]{structure}
\setbeamertemplate{blocks}[rounded]%[shadow=true]
\setbeamercolor{block title}{bg=UBCblue, fg=white}
\setbeamercolor{block body}{bg=UBCblue!20}
\setbeamercolor{block title example}{bg=UBCgreen, fg=white}
\setbeamercolor{block body example}{bg=UBCgreen!20}

% https://tex.stackexchange.com/a/306662
\makeatletter
\def\beamer@framenotesbegin{% at beginning of slide
     \usebeamercolor[fg]{normal text}
      \gdef\beamer@noteitems{}%
      \gdef\beamer@notes{}%
}
\makeatother

\usetheme{Dresden}
\useoutertheme{miniframes} % Alternatively: miniframes, infolines, split
\useinnertheme{circles}
\usefonttheme{professionalfonts}

\title{The consistency of New Foundations}
\subtitle{An alternative foundation for set theory}
\author{Sky Wilshaw, Ya\"el Dillies}
\institute{University of Cambridge}
\date{10th October 2022}

\begin{document}

\begin{frame}
    \titlepage
\end{frame}

\begin{frame}{Outline}
    \tableofcontents[hideallsubsections]
\end{frame}

\section{What is NF?}

\begin{frame}{Set theory and axioms}
    Set theory is used as a foundation for modern mathematics.

    \medskip

    We need to be careful about what we allow to be a set, otherwise we can get paradoxes.

    \[ R := \qty{x\mid x\not\in x};\quad R \in R \iff R \not\in R \]
\end{frame}
\begin{frame}{Zermelo-Fr\"ankel set theory}
    ZF(C) avoids this paradox by restricting set comprehension to only allow us to specify subsets of a set that already exists.

    \[ \qty{x\mid x \not\in x} \text{ is not well-formed};\quad \qty{x\in \mathcal U\mid x \not\in x} \text{ is well-formed} \]

    Russell's paradox no longer works:

    \[ R := \qty{x\in\mathcal U\mid x \not\in x} \implies R \not\in \mathcal U \]
\end{frame}
\begin{frame}{A different fix}
    Perhaps the real problem with the Russell class is not that we are quantifying over all sets, but that the formula \( x \not\in x \) fundamentally does not make sense.

    \medskip

    Quine (1937) suggested we might be allowed to quantify over all sets without leading to contradictions if we are only allowed to use certain formulas.
\end{frame}
\begin{frame}{Stratification}
    \begin{definition}
        A formula \( \phi \) is \emph{stratified} if we can annotate each variable in \( \phi \) with an index in \( \mathbb N \) such that
        \begin{itemize}
            \item if \( x = y \) appears in \( \phi \), the indices on \( x \) and \( y \) are the same; and
            \item if \( x \in y \) appears in \( \phi \), the index on \( y \) is one higher than the index on \( x \).
        \end{itemize}
    \end{definition}
    We will annotate indices with superscripts, so we can write
    \[ x^n = y^n;\quad x^n \in y^{n+1} \]
\end{frame}
% Each index refers to a "universe" of sets at level n
\begin{frame}{Examples}
    These formulas are stratified:
    \begin{enumerate}
        \item \( x^1 \in y^2 \wedge y^2 \in z^3 \wedge z^3 = w^3 \)
        \item extensionality: \( \forall x^2.\, \forall y^2.\, (\forall z^1.\, z^1 \in x^2 \iff z^1 \in y^2) \implies x^2 = y^2 \)
        \item power set: \( \forall x^2.\, \exists y^3.\, \forall z^2.\, (\forall a^1.\, a^1 \in z^2 \implies a^1 \in x^2) \implies z^2 \in y^3 \)
    \end{enumerate}
    These formulas are not stratified:
    \begin{enumerate}
        \setcounter{enumi}{3}
        \item \( x \in x \)
        \item \( x = y \wedge x \in y \)
    \end{enumerate}
    NF is the theory with the axioms of ZF, except set comprehension is allowed over the whole universe, but restricted to stratified formulas.
\end{frame}

% We don't *need* to stratify each formula to use it, or even use a consistent stratification across formulae, we just need to show it can be done.

\begin{frame}{The universe}
    Most things that work in ZF also work in NF, because most of the formulas we care about are stratified, but there are a few exceptions.

    \medskip

    The formula \( x^1 = x^1 \) is stratified, so the universe set \( \qty{x\mid x^1 = x^1} \) exists. However, the Russell class doesn't exist, because the formula \( x \not\in x \) is unstratified.

    \medskip

    For a set \( x \), the formula \( y^1 \not\in x^2 \) is stratified, so the universal set complement \( \qty{y\mid y^1 \not\in x^2} \) exists. In fact, NF forms a Boolean algebra.
\end{frame}
\iffalse
\begin{frame}{Cardinals}
    Cardinals in ZF are equivalence classes of equinumerous sets.
    \[ 1 \text{ is the class of sets } x \text{ where } x \simeq \qty{\varnothing} \]
    In NF, cardinals can be represented as the set of all sets of a given size.
    \[ 1 = \qty{x\mid \exists y^1.\, y^1 \in x^2 \wedge \forall z^1 \in x^2.\, z^1 = y^1} \]
\end{frame}
\begin{frame}{Choice}
    NF disproves the axiom of choice.

    \medskip

    However, a weakened form of the axiom of choice (for example, one that only operates on well-founded sets) may still work well with NF.
\end{frame}
\fi

\section{Our goal}

\begin{frame}{Is NF consistent?}
    We say that NF is \emph{consistent} if there is no formula \( \phi \) such that \( \phi \) is true and \( \neg\phi \) is also true; i.e.\, there are no contradictions.

    Holmes (2010) proved that NF is consistent.

    \begin{proof}[Proof sketch]
        NF is consistent if and only if \emph{tangled type theory} (TTT) is consistent.
        We can build a model of TTT, hence TTT is consistent.
    \end{proof}

    The proof that we can build a model of TTT is difficult to understand and so it is difficult to trust. Our goal is to \emph{formalise} this part of the proof.
    % Yaël, hopefully you can write some stuff to follow!
    % I'm not sure how long this is all going to take to present, I may cut out significant chunks of the above - I'll do some practice over the next few days.
\end{frame}

% might want to say things like what is lean, why can we trust it, what are our results, where did we get stuck?

% I should probably write some kind of conclusion like "further work needed, no definitive conclusion"

\end{document}
